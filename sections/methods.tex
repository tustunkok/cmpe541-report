\section{Methods}
In this section, a summary of the literature on the subject is given. First, a
generic transşation process is described. Then, more specific studies are
presented.

\subsection{A Generic Translation Process From XSD to JSON Schema}
JSON is becoming more and more popular as a data interchange format, but not as
common as XML, which has recently risen in web service applications. However,
converting the XML data type directly to the JSON data type results in a format
error in the original data. One can make the XML-to-JSON conversion error-free
by converting the XML Schema Definition (XSD) to the appropriate JSON Schema.
According to the study of Guo et. al. \autocite{Guo2017}, XSD is deployed to
create a dictionary that converts each XSD node to its corresponding JSON Schema
format. It is found that this method produces good translation results in most
cases.

According to the study, there are several semantic principles must be followed
to transform an XSD snippet into the corresponding JSON Schema. First, a list of
data values is provided. Secondly, the form can only have a maximum of 5 values.
Lastly, each value is an integer that is not negative. Furthermore, a genuine
XML document requires a root element, and this element cannot be used to build
one. A basic and acceptable JSON Schema can be created based on these
assumptions. For the translation process, the procedures are followed as: For
each XSD node, a translation dictionary is created to its matching JSON Schema.
When traversing the XSD tree, the node's id and attribute are initially saved
until there are no more child nodes. Then, using the dictionary, the node is
translated from bottom to top. The data pieces now can be traversed into the
appropriate JSON Schema format. Recursion can be used to explore some XSD
elements with child nodes. In the process of navigating, there are three points
to which need attention: first, each node is given a unique identity (ID), and
its namespace and tag name are saved; second each XSD attribute's keyword names,
values, related collections, and default inheritance attributes are recognized;
third, identifying the element, store its parent node element, and identifying
the textual information if the child element of an element is a simple text and
there is no nested XML node. The element node can be transformed into its
equivalent JSON data type if it has an XSD namespace and is a basic XSD data
type. There are several specified data types in XML Schema. Although JSON has a
restricted amount of data types, XSD can be used to confine these data types to
these limitations. The XSD-to-JSON Schema type-conversion data dictionary is
created by giving an equivalent JSON Schema for all declared XML data types. The
complete translation process is traversed from the XSD node's root node to the
smallest segment (child node), and the child node is then translated. As a
result, the parent node gets translated, and the entire XSD document is finally
converted to JSON Schema format.

Overall, by traversing the XSD document node and defining the data attribute
dictionary for XSD to JSON Schema conversion, the technique may transform XSD
format data into JSON Schema. Because JSON Schema provides fewer data
characteristics, it's difficult to implement all the data dictionary's
conversion rules, making it impossible to convert all XSD parts.

\subsection{Existing Methods}
The literature review on the subject yields two studies that are directly
related to the scope of this report. They are briefly explained in this section.

\subsubsection{From XML Schema to JSON Schema: Translation with CHR}
Again, since it seems impossible to convert JSON and XML formats to each other
without error, even though, Schema languages exist to specify the structure of
sample documents for both data formats, Nogatz et. al. \autocite{Nogatz2014}
introduce an application of a language translator. It uses Prolog and Constraint
Handling Rules (CHR) to transform an XML Schema into a JSON Schema document.
Simply, an XML Schema document is spread out into CHR constraints, which makes
it possible to specify translation rules declaratively.

Nogatz identifies CHR as a declarative programming language that follows the
constraint-based programming paradigm as an extension of the host language. One
can define CHR rules to manipulate entries in a constraint store. That is,
depending on the currently saved constraint, you can remove some or add a new
constraint. Basically, the CHR extension adds three types of rules to the
logical programming construct to control the contents of the constraint store.
These CHR program rules are well known in the mathematical logic of computer
programming. In addition to simplification and propagation rules, there is a
mixed type called \emph{simpagation}. 

The translation process represented in the study has several steps. The first
step is reading the XML Schema into Prolog, the output is a nested Prolog term.
Then, the related constraints will be propagated by traversing the nested Prolog
term recursively. Their unique identities and references to parent and child
nodes keep their places. After that, the defaults should be added since the
Prolog’s XML parser will read explicitly set attributes only. With a CHR
simpagation rule, some settings are also done, too. Then, the translation step
comes. By following the constraints propagated referred in the study,
translation from XSD fragments to JSON Schema is done. All have the same form. A
single JSON constraint is conveyed that holds the XSD fragment's JSON Schema;
the guard guarantees that all node and node attribute constraints are of the XML
Schema namespace. There may be numerous JSON constraints with the same
identifier since some node restrictions apply to multiple CHR rules. A
simpagation rule merges them with the assistance of a self-defined merge JSON
Prolog predicate. The last two steps include wrapping up the JSON Schema and
cleaning. When the root element of the provided XSD has been translated and its
JSON constraints have been combined, the previous phase ends. In addition, the
globally declared type definitions are merged into the root's JSON constraint's
definitions object. Finally, the constructed JSON Schema object is cleaned up.

Overall, the study proposes a language translator was developed to translate an
XML Schema to a JSON Schema equivalent. Because the module was built from the
bottom-up approach, it has a test framework and several test cases. It may not
be feasible to implement all restricting semantics due to the unavailability of
specific capabilities in JSON Schema. Another functionality that is lacking is
the ability to handle linked XSD documents.

\subsubsection{Towards Efficient and Unified XML/JSON Conversion}
As we have already said, for data representation and/or exchange XML and JSON
are the two most widely used formats When switching from one format to another
for data representation, available converters can be used. Šandrih et. al.
\autocite{Sandrih2017} describes a structure that converts XML and JSON formats
to each other, which is compared to an existing and publicly available
converter. Some differences and difficulties are found in the conversion when
using different converters. Some results of the comparison are presented.

JavaScript is used to create the converter. Building a JavaScript object and
deserializing it are the first two steps to converting from XML to JSON for data
representation. The types of the child nodes of the input XML DOM tree of the
document are inspected during the process of building JavaScript objects, which
includes the types of elements, text, and CDATA sections. If a node is an
element, its attributes are initially gathered as fields of the JavaScript
object that corresponds to it. After that, the offspring of this element's
simple fixture is checked. Objects are constructed in somewhat various ways
depending on the quantity of text content and CDATA sections. If an element is
not a leaf, the same function is run recursively for the subtree with this child
element as the root. Instead of using a built-in function, the deserialization
step is done with a custom function. Steps in the conversion of input JSON to
the appropriate XML representation are converter independent. To parse a generic
JSON string: First, the open tag () is written first, followed by the
attributes-values pairs; if this element has children, the close tag (>) is
inserted next, and the function is performed recursively using the children
elements as arguments. The text and CDATA parts are left alone. The parent
element's tag closes after a recursive call. If an element has no children, it
is closed immediately (/>).

Also, two converters are compared in this study.  In several cases, it is
discovered that both converters suggest different conversions. These two options
are available, based on personal preferences and logic. The following are some
possible options: If CDATA sections are present in data and the content of these
sections is relevant, one converter may be a better option. With one converter,
data loss is conceivable. Similarly, with one converter, time climbs rapidly for
huge files. Invertibility might be an essential criterion as well. A solution is
offered in this study that would result in preserving most of the information.
It is managed to save the content, but in some cases, the order could not be
restored, which is left as future work.