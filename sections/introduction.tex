\section{Introduction}
Today, almost every web application is working asynchronously. While a
long-running user task is running in the background, the web application shows a
waiting animation to the user. Asynchronous web applications immensely increase
user satisfaction and provide informative feedback to the user.

Getting the necessary information to start a background task and informing the
user that the background process finished its job is made possible by sending
and receiving messages between server and client. These messages should have
well-defined structures and standards. They should also structure the
transmitted data and store it when necessary. These messages can be plain text
as well as proprietary binary data. Both have their advantages and
disadvantages. Employing plain text messages over HTTP has its advantages
against using binary data. Because of that, text-based markup languages are
developed and have quickly become popular. Two of the most famous ones are
Extensible Markup Language (XML) and JavaScript Object Notation (JSON). Since
these languages have similar/interchangeable tasks, converting one to another
can be very useful in some contexts.

\subsection{Motivation}
XML Schema Definition (XSD) is a very stable and widespread set of elements to
be used with XML to transfer data back and forth across the server and host. W3C
published its recommendation of XSD in 2001. Since then, it has been maintained
and protected by various standards.

The history of JSON Schema is quite similar to the one of XSD, with one key
difference. The syntax of JSON is immensely influenced by the JavaScript
dictionary notation. This property is very web developer-friendly since they are
often familiar with JavaScript. In addition, JSON was developed mainly
considering web technologies such as HTTP, Flash, and JavaScript.

All of these notches suggest some sort of conversion should be made possible
from XSD to JSON. However, the literature has little research on the topic.
Therefore, a summary of the existing studies can be advantageous to have for
developing further research ideas. Besides, due to the popularity of XSD, some
JSON suitable data have already been stored in XSD rather than JSON. The
conversion of this type of data can be beneficial. Also, transmitting messages
with JSON is much more optimized in some modern environments.

\subsection{Preliminaries}
In general, a schema describes the characteristics of an object and its
relations with other objects. A schema defines elements and attributes of
elements to represent an object such as a web page. Examples of such a schema
can be HTML, XML. A schema does not have to be a document. It can be a byte
stream or a database record. In this report, a schema is always interpreted as a
document.

To understand the actual need for conversion, one should understand the
underlying formats. In this section, a summary of both XSD and JSON Schema is
presented. The explanations for each format are also supported with brief
examples.

\subsubsection{XML Schema Definition}
Fallside et al. \autocite{Fallside} start giving an example XML document (but
not a schema yet) on a home products ordering and billing application in their
primer work on XML Schema. Their example document can be seen in Listing \ref{lst:xml-doc}.

\begin{lstlisting}[language = XML, caption = {An XML document for a home products ordering and billing application.}, captionpos = b, label = lst:xml-doc]
    <?xml version="1.0"?>
    <purchaseOrder orderDate="1999-10-20">
    <shipTo country="US">
        <name>Alice Smith</name>
        <street>123 Maple Street</street>
        <city>Mill Valley</city>
        <state>CA</state>
        <zip>90952</zip>
    </shipTo>
    <billTo country="US">
        <name>Robert Smith</name>
        <street>8 Oak Avenue</street>
        <city>Old Town</city>
        <state>PA</state>
        <zip>95819</zip>
    </billTo>
    <comment>Hurry, my lawn is going wild<!/comment>
    <items>
        <item partNum="872-AA">
            <productName>Lawnmower</productName>
            <quantity>1</quantity>
            <USPrice>148.95</USPrice>
            <comment>Confirm this is electric</comment>
        </item>
        <item partNum="926-AA">
            <productName>Baby Monitor</productName>
            <quantity>1</quantity>
            <USPrice>39.98</USPrice>
            <shipDate>1999-05-21</shipDate>
        </item>
    </items>
    </purchaseOrder>
\end{lstlisting}

Listing 1 starts with a version identification, continues with several
subelements (elements that contain other elements) that are spanned by a
top-level element called "purchaseOrder". Each element has its closing tag if
they are not one-liners. Elements that carry subelements or attributes are
called complex types. Other elements that only have numbers, strings, dates,
etc. are called simple types.

An XML document is said to be an XML schema when its elements and subelements
have "xsd" namespace associated with them. A partial example from Fallside et
al. is given in Listing \ref{lst:xml-sch}.

\begin{lstlisting}[language = XML, caption = {A partial XML schema for a home products ordering and billing application.}, captionpos = b, label = lst:xml-sch]
    <xsd:schema xmlns:xsd="http://www.w3.org/2001/XMLSchema">

    <xsd:annotation>
        <xsd:documentation xml:lang="en">
        Purchase order schema for Example.com.
        Copyright 2000 Example.com. All rights reserved.
        </xsd:documentation>
    </xsd:annotation>

    <xsd:element name="purchaseOrder" type="PurchaseOrderType"/>

    <xsd:element name="comment" type="xsd:string"/>

    <xsd:complexType name="PurchaseOrderType">
        <xsd:sequence>
        <xsd:element name="shipTo" type="USAddress"/>
        <xsd:element name="billTo" type="USAddress"/>
        <xsd:element ref="comment" minOccurs="0"/>
        <xsd:element name="items"  type="Items"/>
        </xsd:sequence>
        <xsd:attribute name="orderDate" type="xsd:date"/>
    </xsd:complexType>

    ...

    </xsd:schema>

\end{lstlisting}

\subsubsection{JSON Schema Definition}